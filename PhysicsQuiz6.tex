\documentclass[12pt, letterpaper, twoside]{article}
\usepackage[utf8]{inputenc}
\usepackage{ mathrsfs }

\usepackage{titlesec}
\usepackage{circuitikz}

\titlespacing\section{0pt}{12pt plus 4pt minus 2pt}{0pt plus 2pt minus 2pt}
\titlespacing\subsection{0pt}{12pt plus 4pt minus 2pt}{0pt plus 2pt minus 2pt}
\titlespacing\subsubsection{0pt}{12pt plus 4pt minus 2pt}{0pt plus 2pt minus 2pt}


\title{Math 260 Notes}
\author{Zachary Chao}
\date{April 03 2018}

%Allows indentation for a whole paragraph
\newenvironment{myindentpar}[1]%
  {\begin{list}{}%
    {\setlength{\leftmargin}{#1}}%
    \item[]%
  }
{\end{list}}
  

\begin{document}

\setcounter{section}{10}
\section{Magnetic Forces and Fields}
  \subsection*{Symbols and Shit}
    \begin{tabular}{c l c l}
  	  \textbf{Symbol} & \textbf{Meaning} & \textbf{Unit Symbol} & \textbf{Units}\\
    \end{tabular}
  
  \subsection*{Equations}
    \begin{tabular}{l c l}
    \end{tabular}
    
    
  \subsection*{Prefixes}
    \begin{tabular}{l l}
  	  \textbf{Symbol} & \textbf{Multiplier}\\
      $G$ & $10^9$\\
      $M$ & $10^6$\\
      $c$ & $10^{-2}$\\
      $m$ & $10^{-3}$\\
      $\mu$ & $10^{-6}$\\
      $n$ & $10^{-9}$\\
      $p$ & $10^{-12}$\\
  \end{tabular}
  
  \setcounter{subsection}{1}
  \subsection{Magnetic Fields and Lines - 15, 17, 19, 23}
  
  
  \subsection{Motion of a Charged Particle in a Magnetic Field - 25, 28}
  
  
  \subsection{Magnetic Force on a Current-Carrying Conductor - 33, 34, 35, 36}
  
  
  \subsection{The Hall Effect - 49}
  
  
  \subsection{Applications of Magnetic Forces and Fields - None}
  
  
  \subsection{Additional Problems - 59, 62, 64, 69, 81}
    

\end{document}