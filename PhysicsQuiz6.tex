\documentclass[12pt, letterpaper, twoside]{article}
\usepackage[utf8]{inputenc}
\usepackage{ mathrsfs }

\usepackage{titlesec}
\usepackage{ upgreek }

\titlespacing\section{0pt}{12pt plus 4pt minus 2pt}{0pt plus 2pt minus 2pt}
\titlespacing\subsection{0pt}{12pt plus 4pt minus 2pt}{0pt plus 2pt minus 2pt}
\titlespacing\subsubsection{0pt}{12pt plus 4pt minus 2pt}{0pt plus 2pt minus 2pt}


\title{Math 260 Notes}
\author{Zachary Chao}
\date{April 03 2018}

%Allows indentation for a whole paragraph
\newenvironment{myindentpar}[1]%
  {\begin{list}{}%
    {\setlength{\leftmargin}{#1}}%
    \item[]%
  }
{\end{list}}
  

\begin{document}

\setcounter{section}{10}
\section{Magnetic Forces and Fields}
  \subsection*{Symbols and Shit}
    \begin{tabular}{c l c l}
  	  \textbf{Symbol} & \textbf{Meaning} & \textbf{Unit Symbol} & \textbf{Units}\\
  	  \textbf{Chapter 11} & & &\\
  	  $T$ & Magnetic Field & N/Am & Newtons per amp meter\\
  	  $Wb$ & Magnetic Flux & $T*m^2$ & Tesla meter squared\\
  	  $I$ & Current & A & amp\\
  	  $l$ & Length lol & m & meters\\
  	  $B$ & Magnetic Field & T & teslas\\
  	  $E$ & Electric Field & kV/m & kiloVolt per meter\\
  	  \textbf{Chapter 12} & & &\\
  	  $N$ & Number of Turns & None & None\\
  	  $J$ & Current Density & $A/m^2$ & amps per meter squared\\
      $q_e$ & Charge of Electron & $C$ & coulomb\\
      $n_e$ & Electron Density & electrons / area & electrons per cubic meter\\
      $v_d$ & Drift Velocity & $m/s^2$ & average speed of electrons\\
    \end{tabular}
  
  \subsection*{Equations}
    \begin{tabular}{l c l}
    \textbf{Chapter 11} & &\\
      $\vec{F} = (qvB sin \theta, RHR)$\\
	  $\vec{F} = (IlB sin \theta, RHR)$\\
      $W = \vec{F} \cdot \vec{d} = 0$\\
      $qvB = \frac{mv^2}{r}$\\
      $eE = ev_dB$\\
      $I = nev_dA$\\
      $V_{hall} = IBl / neA$\\
      $E = Bc$\\
      \textbf{Chapter 12} & &\\
      $\vec{B} = (\frac{\mu_0}{4 \pi} \frac{qv sin \theta}{r^2}, RHR)$\\
      $d \vec{B} = \frac{\mu_0}{4 \pi} \frac{Idl sin \theta}{r^2}$\\
      $\vec{E} = \frac{1}{4 \pi \upvarepsilon_0} \frac{q}{r^2}$\\
      $B_{infLine} = \frac{\mu_0 I}{2 \pi r}$\\
      $B_{centerLoop} = \frac{\mu_0 I \phi}{4 \pi R}$\\
      $F = \frac{\mu_0 I_1 I_2 l}{2 \pi r}$\\
      $\int_{line} Bdl \cos \theta$\\
      $\oint Bdl \cos \theta = \mu_0 I_{encl}$\\
      $\vec{B} = \frac{\mu_0I_0N}{2L}(\sin \theta_2 - \sin \theta_1)$\\
      $\vec{B}_{toroid} = \frac{\mu_0I_0N}{2 \pi r}$\\
      
    \end{tabular}
    
    
  \subsection*{Prefixes}
    \begin{tabular}{l l}
  	  \textbf{Symbol} & \textbf{Multiplier}\\
      $G$ & $10^9$\\
      $M$ & $10^6$\\
      $c$ & $10^{-2}$\\
      $m$ & $10^{-3}$\\
      $\mu$ & $10^{-6}$\\
      $n$ & $10^{-9}$\\
      $p$ & $10^{-12}$\\
  \end{tabular}
  
  \subsection*{Constants}
    \begin{tabular}{l l}
      \textbf{Thing} & \textbf{Value}\\
      $Q_e$ & $1.602 \times 10^{-19}C$\\
      $m_e$ & $9.109 \times 10^{-31}kg$\\
      $\mu_0$ & $4 \pi \times 10^{-7}N/A^ 2$
    
    \end{tabular}
  
  \setcounter{subsection}{1}
  \subsection{Magnetic Fields and Lines - 15, 17, 19, 23}
  
  
  \subsection{Motion of a Charged Particle in a Magnetic Field - 25, 28}
  
  
  \subsection{Magnetic Force on a Current-Carrying Conductor - 33, 34, 35, 36}
  
  
  \subsection{The Hall Effect - 49}
  
  
  \subsection{Applications of Magnetic Forces and Fields - None}
  
  
  \subsection{Additional Problems - 59, 62, 64, 69, 81}
    
\section{Chapter }

  \subsection{The Biot-Savart Law - 17, 18, 19, 21}
  
  \subsection{Magnetic Field Due to a Thin Straight Wire - 25, 27, 29}

  \subsection{Magnetic Force between Two Parallel Currents - 31, 33, 34}
  
  \subsection{Magnetic Field of a Current Loop - 38}
  
  \subsection{Amperes Law - 45, 46, 47, 48}
  
  \subsection{Solenoids and Toroids - 52, 55, 56}
    \subsubsection*{Question 52}
	  $$\vec{B} = \frac{\mu_0I_0N}{2L}(\sin \theta_2 - \sin \theta_1)$$
	  First theta is angle between position and bottom of solenoid, second theta is angle between position and top of solenoid.
      
  
  \subsection{Magnetism in Matter - None}
  
  \subsection*{Additional Problems - 67, 80, 85}

\end{document}