\documentclass[12pt, letterpaper, twoside]{article}
\usepackage[utf8]{inputenc}
\usepackage{ mathrsfs }
\usepackage{amsmath}


\title{Linear Algebra Test 2 Study Guide}
\author{Zachary Chao}
\date{April 10 2018}
 
\newcommand{\real}{\rm I\!R}
\newcommand{\domain}{\mathscr{D}}

\begin{document}

\maketitle
Test from Chapter 3 to Chapter 4.7

\setcounter{section}{2}
\section{Determinants}
\subsection{The Determinant of a Matrix}
\begin{enumerate}
\item Find the determinant of a 2x2 matrix
\item Find the minors and cofactors of a matrix
\item Use expansion by cofactors to find the determinant of a matrix
\item Find the determinant of a triangular matrix
\end{enumerate}

Only square matrices have determinants.\\

\subsubsection*{Definition of Minors and Cofactors of a Matrix}
The minor $M_{ij}$ of the element $a_{ij}$ is the determinant of the matrix obtained by deleting the $i^{th}$ row and the $j^{th}$ column of A. The cofactor $C_{ij}$ is given by $C_{ij} = (-1)^{i+j}M_{ij}$.\\

\subsubsection*{Expansion by cofactors}
$$det(A) = \sum_{j=1}^n a_{ij} C_{ij}$$

\subsection{Determinants and Elementary Operations}
\begin{enumerate}
\item Use elementary row operations to evaluate a determinant
\item Use elementary column operations to evaluate a determinant
\item Recognize conditions that yield zero determinants
\end{enumerate}

\subsubsection*{Definition of a Triangular Matrix}
A triangular matrix is a matrix in which all entries above or below the main diagonal are zero. For example - \\


\[
\begin{bmatrix}
 a & 0 & 0 & 0\\
 e & f & 0 & 0\\
 i & j & k & 0\\
 m & n & o & p
\end{bmatrix}
or 
\begin{bmatrix}
 a & b & c & d\\
 0 & f & g & h\\
 0 & 0 & k & l\\
 0 & 0 & 0 & p
\end{bmatrix}
\]


\subsubsection*{Determinant of a Triangular Matrix}
If $A$ is a triangular matrix of order $n$, then its determinant is the product of the entries on the main diagonal. 
$$det(A) = a_{11}a_{22}a_{33} \dots a_{nn}$$

\subsubsection*{Elementary Row Operations and Determinants}
\begin{enumerate}
\item When B is obtained from A by interchanging two rows of A, $det(B) = -det(A)$
\item When B is obtained from A by adding a multiple of a row A to another row of A $det(B) = det(A)$
\item when B is obtained from A by multiplying a row of A by a nonzero constant c, $det(B) = c det(A)$
\end{enumerate}
\textbf{Note} These all hold true for columns as well.

\subsubsection*{Conditions That Yield a Zero Determinant}
\begin{enumerate}
\item An entire row or column consists of only zeros
\item Two rows or columns are equal
\item One row or column is a multiple of another row or column
\end{enumerate}

\subsection{Properties of Determinants}
\begin{enumerate}
\item Find the determinant of a matrix product and scalar multiple of a matrix
\item Find the determinant of an inverse matrix and recognize equivalent conditions for a nonsingular matrix
\item Find the determinant of a transpose of a matrix
\end{enumerate}

\subsubsection*{Determinant of a Matrix Product}
If $A$ and $B$ are square matrices of order $n$, then
$$det(AB) = det(A)det(B)$$

\subsubsection*{Determinant of a Scalar Multiple of a Matrix}
If $A$ is a square matrix of order $n$ and $c$ is a scalar, then the determinant of $cA$ is 
$$det(cA) = c^ndet(A)$$

\subsubsection*{Determinant of an Invertible Matrix}
A square matrix $A$ is invertible (nonsingular) if and only if $det(A) \neq 0$

\subsubsection*{Determinant of an Inverse Matrix}
If $A$ is an $nxn$ invertible matrix, then 
$$det(A^{-1}) = \frac{1}{det(A)}$$

\subsubsection*{Equivalent Conditions for a Nonsingular Matrix}
If A is a $nxn$ nonsingular (invertible) matrix, then the following statements are equivalent.
\begin{enumerate}
\item $A$ is invertible
\item $Ax=b$ has a unique solution for every $nx1$ column matrix
\item $Ax=0$ has only the trivial solution
\item $A$ is row-equivalent to $I_n$
\item A can be written as the product of elementary matrices
\item $det(A) \neq 0$
\end{enumerate}

\subsubsection*{Determinant of a Transpose}
if $A$ is a square matrix, then 
$$det(A^T) = det(A)$$

\subsection{Applications of Determinants}
\begin{enumerate}
\item Use Cramer's rule to solve a system of $n$ linear equations
\item Use determinants to find the area, volume and the equations of lines and planes
\end{enumerate}

\subsubsection*{Cramer's Rule}
If a system of $n$ linear equations in $n$ variables has a coefficient matrix $A$ with a nonzero determinant $|A|$, then the solution of the system is 
$$x_1 = \frac{|A_1|}{|A|}, x_2 = \frac{|A_2|}{|A|}, \dots, x_n = \frac{|A_n|}{|A|}$$
Where $A_1$ is created by replacing the first column with the $b$ column.

\subsubsection*{Area of a Triangle in the xy-Plane}
The area of a triangle with vertices $(x_1,y_1), (x_2,y_2)$ and $(x_3, y_3)$ is 

$$
A = \pm \frac{1}{2} det
\begin{bmatrix}
 x_1 & y_1 & 1\\
 x_2 & y_2 & 1\\
 x_3 & y_3 & 1
\end{bmatrix}
$$

\subsubsection*{Test For Collinear Points in the xy-Plane}
Three points $(x_1,y_1), (x_2,y_2)$ and $(x_3, y_3)$ are collinear if and only if 
$$
\begin{bmatrix}
 x_1 & y_1 & 1\\
 x_2 & y_2 & 1\\
 x_3 & y_3 & 1
\end{bmatrix}
= 0
$$

\subsubsection*{Two-Point Form of the Equation of a Line}
An equation of the line passing through the distinct points $(x_1,y_1)$ and $(x_2,y_2)$ is given by 
$$
\begin{bmatrix}
 x & y & 1\\
 x_1 & y_1 & 1\\
 x_2 & y_2 & 1
\end{bmatrix}
= 0
$$

\subsubsection*{Volume of a Tetrahedron}
The volume of a tetrahedron with vertices $(x_1,y_1), (x_2,y_2) (x_3, y_3)$ and $(x_4, y_4)$ is 
$$
V = \pm \frac{1}{6}
\begin{vmatrix}
 x_1 & y_1 & z_1 & 1\\
 x_2 & y_2 & z_2 & 1\\
 x_3 & y_3 & z_3 & 1\\
 x_4 & y_4 & z_4 & 1
\end{vmatrix}
$$










\end{document}



