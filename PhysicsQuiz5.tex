\documentclass[12pt, letterpaper, twoside]{article}
\usepackage[utf8]{inputenc}
\usepackage{ mathrsfs }

\usepackage{titlesec}
\usepackage{circuitikz}

\titlespacing\section{0pt}{12pt plus 4pt minus 2pt}{0pt plus 2pt minus 2pt}
\titlespacing\subsection{0pt}{12pt plus 4pt minus 2pt}{0pt plus 2pt minus 2pt}
\titlespacing\subsubsection{0pt}{12pt plus 4pt minus 2pt}{0pt plus 2pt minus 2pt}


\title{Math 260 Notes}
\author{Zachary Chao}
\date{April 03 2018}

%Allows indentation for a whole paragraph
\newenvironment{myindentpar}[1]%
  {\begin{list}{}%
    {\setlength{\leftmargin}{#1}}%
    \item[]%
  }
{\end{list}}
  

\begin{document}

\setcounter{section}{8}
\section{Current and Resistance}
  \subsection*{Symbols and Shit}
    \begin{tabular}{c l c l}
  	  \textbf{Symbol} & \textbf{Meaning} & \textbf{Unit Symbol} & \textbf{Units}\\
      $I$ & Current & $A$ & amps\\
      $\sigma$ & Conductivity & $\frac{1}{\Omega m}$ & inverse ohm-meter\\
      $\rho$ & Resistivity & $\Omega m$ & ohm-meter\\
      $Q$ & Charge & $C$ & coulombs\\
      $P$ & Power & $W$ & watts\\
      $W$ & Work & $J$ & joules\\
      $J$ & Current Density & $A/m^2$ & amps per meter squared\\
      $q_e$ & Charge of Electron & $C$ & coulomb\\
      $n_e$ & Electron Density & electrons / area & electrons per cubic meter\\
      $A$ & Area & $m^2$ & meters squared\\
      $v_d$ & Drift Velocity & $m/s^2$ & average speed of electrons\\
      $E$ & Electric Field & $N/C$ & Newtons per Coulomb\\
      $\varepsilon$ & EMF & $V$ & Volts\\
    \end{tabular}
  
  \subsection*{Equations}
    \begin{tabular}{l c l}
      $P_{supplied}$ & $=$ & $IV$\\
      $P_{dissipated}$ & $=$ & $I^2R$\\
      $P$ & $=$ & $\frac{V^2}{R}$\\
      $V$ & $=$ & $IR$\\
      $R$ & $=$ & $\rho L / A$\\
      $\sigma$ & $=$ & $1/\rho$\\
      $I$ & $=$ & $q_en_eAv_d$\\
      $J$ & $=$ & $\sigma E$\\
      $J$ & $=$ & $I/A$\\
    \end{tabular}
    
    
  \subsection*{Prefixes}
    \begin{tabular}{l l}
  	  \textbf{Symbol} & \textbf{Multiplier}\\
      $G$ & $10^9$\\
      $M$ & $10^6$\\
      $c$ & $10^{-2}$\\
      $m$ & $10^{-3}$\\
      $\mu$ & $10^{-6}$\\
      $n$ & $10^{-9}$\\
      $p$ & $10^{-12}$\\
      
    
  \end{tabular}
    
  \subsection{Electrical Current - 23}
    \subsubsection*{Question 23}
      \begin{description}
        \item $Q_{electron} = 1.6x10^{-19}C$
        \item $I = \Delta Q / \Delta t$
      \end{description}      
  \subsection{Model of Conduction in Metals - 30}
    \subsubsection*{Question 30}
      \begin{description}
        \item \textbf{a)} $J = I/A$
      \end{description}
  \subsection{Resistivity and Resistance - 39, 41, 42}
    \subsubsection*{Question 39}
      \begin{description}
        \item $R = \rho L/A$
        \item $\rho_{W} = 5.6$ x $10^{-8} \ \Omega m$
      \end{description}
    \subsubsection*{Question 41}
      \begin{description}
        \item $R = \rho L / A$   
        \item $\rho_{Cu} = 1.72$ x $10^{-8} \ \Omega m$
        \item $\rho_{Al} = 2.65$ x $10^{-8} \ \Omega m$
      \end{description}
    \subsubsection*{Question 42}
      \begin{description}
        \item $R = \rho L / A$
        \item $I = V / R$
        \item $\rho_{Silicon} = 6.4$ x $10^2 \ \Omega m$
      \end{description}
  \subsection{Ohm's Law - None}
    Fuck Ohm's Law.
  \subsection{Electrical Energy and Power - 57, 61}
    \subsubsection*{Question 57}
      \begin{description}
        \item 14-guage means diameter = $1.63mm$
        \item $\rho_{Nichrome} = 100$ x $10^{-8} \ \Omega m$
        \item $I = P/V$
        \item $R = P/I^2$
        \item $L = RA/\rho$
      \end{description}
  \subsection{Superconductors - 63}
    \subsubsection*{Question 63}
      \begin{description}
        \item $\rho_{Cu} = 1.68$ x $10^{-8} \ \Omega m$
        \item The equation $P = I^2R$ gives us the amount of power dissipated through a medium. A superconductor has the property that $\sigma = \infty$. As we know that $\sigma = 1/R$, $R = 0$. So the power dissipated will also be 0.
        \item $R = \rho L / A$
        \item $P = I^2R$
      \end{description}
  \subsection*{Additional Problems - 73, 77}
    \subsubsection*{Question 73}
      $J = I/A$\\
      $R = V/I$\\
      $\rho = RA/L$
      
\section{Direct-Current Circuits} 
  \begin{tabular}{c c c}
  	   & \textbf{Series} & \textbf{Parallel}\\
      \textbf{I} \quad & $I = I_1 = I_2 = I_3 \quad$ & $I = I_1 + I_2 + I_3 \quad$\\
      \textbf{V} \quad & $V = V_1 + V_2 + V_3 \quad$ & $V = V_1 = V_2 = V_3 \quad$\\
      \textbf{R} \quad & $R = R_1 + R_2 + R_3 \quad$ & $\frac{1}{R} = \frac{1}{R_1} + \frac{1}{R_2} + \frac{1}{R_3}$\\
      \textbf{Q} \quad & $Q = Q_1 = Q_2 = Q_3 \quad$ & $Q = Q_1 + Q_2 + Q_3$\\
      \textbf{C} \quad & $\frac{1}{C} = \frac{1}{C_1} + \frac{1}{C_2} + \frac{1}{C_3} \quad$ & $C = C_1 = C_2 = C_3$\\
    \end{tabular}
  \subsection{Electromotive Force - None}
  \subsection{Resistors in Series and Parallel - 26, 27, 28, 30, 31, 33}
  	\subsubsection*{Question 26}
  	  $$R_{series} = \sum R_n$$
  	  $$\frac{1}{R_{parralel}} = \sum \frac{1}{R_n}$$
  	\subsubsection*{Question 27}
  	  Largest = series\\
  	  Smallest = parallel
  	\subsubsection*{Question 28}
  	  $V_{parallel} = V_1 = V_2 = V_3 \quad$\\
  	  $P = IV$
  	\subsubsection*{Question 30}
  	  $R_{series} = R_1 + R_2 + R_3$\\
  	  $R = IV$\\
  	  $P = IV$\\
  	  \\
  	  $\frac{1}{R_{parallel}} = \frac{1}{R_1} + \frac{1}{R_2} + \frac{1}{R_3}$\\
  \subsection{Kirchoff's Rules - 36, 37, 38, 39, 40, 43}
    \begin{circuitikz} \draw
	  (0,1) to[battery1] (0,-1)
	  ;
	\end{circuitikz}
	Current goes from negative to positive ends of the battery, in the photo it will go from bottom to top.\\
	With the current, negative voltage.\\
	Against the current, positive voltage.\\
	With a battery, positive voltage.\\
	Against a battery, negative voltage.	  
  \subsection{Electrical Measuring Instruments - None}
  \subsection{RC Circuits - 51, 52, 53}
    \subsubsection*{Question 51}
      $0.1 \mu F = \tau$\\
      $\tau = RC$
  \subsection{Household Wiring and Electrical Safety - None}
  \subsection*{Additional Problems - 72, 77, 78, 84, 85}


\end{document}