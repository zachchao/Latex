\documentclass[12pt, letterpaper, twoside]{article}
\usepackage[utf8]{inputenc}
\usepackage{ mathrsfs }


\title{Math 260 Notes}
\author{Zachary Chao}
\date{April 03 2018}
 
\newcommand{\real}{\rm I\!R}
\newcommand{\domain}{\mathscr{D}}
\newcommand{\laplace}{\mathcal{L}}
\usepackage{amsmath}



\begin{document}

\section{Impulse Functions (Delta Functions)}
$$ay'' + by' + cy = g(t)$$
When $g(t)$ is large and over a short time interval
$$t_0 - \tau < t < t_0 + \tau$$


\[
  d_\tau(t) = g(t) = 
  \begin{cases}
  	\frac{1}{2\tau} & -\tau < t < \tau \\
  	0 & everywhere \ else \\
  \end{cases}
\]
$I(\tau) = \int_{t_0 - \tau}^{t_0 + \tau} g(t) dt$\\
$I(\tau) = \int_{-\infty}^{\infty} g(t) dt$\\
\[
  Dirac \ Delta \ Function
  \begin{cases}
  	\delta(t) = 0 & t \neq 0 \\
  	\int_{-\infty}^{\infty} \delta(t) dt = 1\\
  \end{cases}
\]
$\delta(t) = \lim_{\tau \to 0} d_\tau(t)$\\
$\int_{-\infty}^\infty \delta(t) dt = \lim_{\tau \to 0} \int_{-\infty}^\infty d_\tau(t) dt$\\
$\delta (t-t_0) = 0, t \neq t_0$\\
$\int_{-\infty}^\infty \delta (t-t_0) dt = 1$\\
How do we take the Laplace Transform of the Dirac Delta Function?\\
\textbf{Define}\\
$\laplace\{\delta (t-t_0)\} = \lim_{\tau \to 0} \delta\{d_\tau(t-t_0)\}$\\



\subsection*{Final Exam}
Tuesday May $22^{nd}$ 4-5:50


\subsection*{Quiz Homework}
5.1 : 21-27 odd\\
5.2 : 1-11 odd, 15, 21a\\
6.1 : 1, 5ab, 7, 11\\




\end{document}



