\documentclass[12pt, letterpaper, twoside]{article}
\usepackage[utf8]{inputenc}
\usepackage{ mathrsfs }


\title{Math 260 Notes}
\author{Zachary Chao}
\date{April 03 2018}

\newcommand{\laplace}{\mathcal{L}}
\usepackage{amsmath}
\usepackage{ upgreek }

\begin{document}
  \setcounter{section}{5}
  \section{The Laplace Transform}
    \textbf{Note : } Laplace table is on page 338.
    \setcounter{subsection}{1}
    \subsection{Solution of Initial Value Problems}
      $6.2 : 1-7odd, 11-15$\\
      \subsection*{Question 1}
        $$\laplace (sin \ at) = \frac{a}{s^2 + a^2}$$
        \textbf{Note :} Constants may be pulled out before the Laplace is applied.
       \subsection*{Question 3}
         $$F(s) = \frac{2}{s^2+3s-4} = \frac{2}{(s-1)(s+4)}$$
         $$\frac{2}{(s-1)(s+4)} = \frac{A}{s-1} + \frac{B}{s+4}$$
         $$2 = A(s+4) + B(s-1)$$
         Now plug values in for s, plugging in 1 and -4 will isolate A or B respectively.\\
         Plugging in $s = 1$ renders us $2 = 5A$, $A = \frac{2}{5}$.\\
         Plugging in $s = -4$ renders us $2 = -5B$, $B = -\frac{2}{5}$.\\
         So, 
         $$\frac{2}{(s-1)(s+4)} = \frac{2}{5(s-1)} - \frac{2}{5(s+4)}$$
         $$\laplace (e^{at}) = \frac{1}{s-a}$$
         
     
    \subsection{•}
      $6.3 : 1,3,5,7,11,13,15,21$\\
      
      
    \section*{Notes}
      \subsection*{Chapter 7 - Systems of First Order Differential Equations}
        \[
        A = 
		\begin{bmatrix}
		  1 & 1\\
		  3 & -1\\
		\end{bmatrix}
		\]
		Example
		\[
        A = 
		\begin{bmatrix}
		  1 & 1\\
		  3 & -1\\
		\end{bmatrix}
		\begin{bmatrix}
		  2\\
		  -6\\
		\end{bmatrix}
		=
		\begin{bmatrix}
		  -4\\
		  12\\
		\end{bmatrix}
		= -2
		\begin{bmatrix}
		  2\\
		  -6\\
		\end{bmatrix}
		\]
		
		$$A \vec{x} = \lambda \vec{x}$$
		$$\lambda = eigenvalue, \vec{x} = eigenvector$$
		For $\lambda = -2$, not only is 
		$\begin{bmatrix}
		  2\\
		  -6\\
		\end{bmatrix}$
		an eigenvector, but so is 
		$\begin{bmatrix}
		  1\\
		  -3\\
		\end{bmatrix}$, 
		$\begin{bmatrix}
		  3\\
		  -9\\
		\end{bmatrix}$
		$\dots$, etc
		
		\[
		\begin{bmatrix}
		  1 & 1\\
		  3 & -1\\
		\end{bmatrix}
		\begin{bmatrix}
		  1\\
		  -3\\
		\end{bmatrix}
		= -2
		\begin{bmatrix}
		  1\\
		  -3\\
		\end{bmatrix}
		\]
		
		A has another eigenvalue, 2. And the corresponding eigenvectors are 
		\[
		\begin{bmatrix}
		  \upvarepsilon_1\\
		  \upvarepsilon_1\\
		\end{bmatrix}
		= \upvarepsilon_1
		\begin{bmatrix}
		  1\\
		  1\\
		\end{bmatrix}
		\]
		
		To find eigenvalues (if they exist) and corresponding eigenvectors,\\
		$\vec{A} \vec{x} = \lambda \vec{x}$\\
		$\vec{A} \vec{x} = \lambda \vec{I} \vec{x}$\\
		$\vec{A} \vec{x} - \lambda \vec{I} \vec{x} = \vec{0}$\\
		$(\vec{A} - \lambda \vec{I})\vec{x} = \vec{0}$\\
		$det(\vec{A} - \lambda \vec{I}) = \vec{0}$\\
		Example : Find the eigenvalues and eigenvectors of 
		$\vec{A} = \begin{bmatrix}
		  1 & 1\\
		  3 & -1\\
		\end{bmatrix}$
		
		$$\vec{A} - \lambda \vec{I} = 
		\begin{bmatrix}
		  1 & 1\\
		  3 & -1\\
		\end{bmatrix}
		-
		\begin{bmatrix}
		  \lambda & 0\\
		  0 & \lambda\\
		\end{bmatrix}$$
		
		$$\vec{A} - \lambda \vec{I} = 
		\begin{bmatrix}
		  1 - \lambda & 1\\
		  3 & -1 - \lambda\\
		\end{bmatrix}$$
		
		$$\begin{vmatrix}
		  1 - \lambda & 1\\
		  3 & -1 - \lambda\\
		\end{vmatrix} = 0$$
		
		$$(1-\lambda)(-1-\lambda)-3 = 0$$
		$$\lambda^2 - 4 = 0$$
		$$\lambda = \pm 2$$
		$$(\vec{A} - \lambda \vec{I}) \vec{\upvarepsilon} = \vec{0}$$
		
		$$
		\begin{bmatrix}
		  1-2 & 1\\
		  3 & -1-2\\
		\end{bmatrix}
		\begin{bmatrix}
		  \upvarepsilon_1\\
		  \upvarepsilon_2\\
		\end{bmatrix}		
		=		
		\begin{bmatrix}
		  0\\
		  0\\
		\end{bmatrix}
		$$
		
		$$
		\begin{bmatrix}
		  -1 & 1\\
		  3 & -3\\
		\end{bmatrix}
		\begin{bmatrix}
		  \upvarepsilon_1\\
		  \upvarepsilon_2\\
		\end{bmatrix}		
		=		
		\begin{bmatrix}
		  0\\
		  0\\
		\end{bmatrix}
		$$	
		
	\subsubsection*{Homework}
	  $7.3 : 16-22$
\end{document}